\chapter{LTHN}

\textbf{Motivação:} $\star\star\star$\\

\textbf{Tipo:} Projeto (que se tornará uma atividade)\\

\textbf{Descrição:} Há anos o Laboratório de Termo-Hidráulica e Neutrônica (\textbf{LTHN}) funciona na metodologia de demanda-solução. Com isso, o parque 
computacional, hoje com 6 servidores e 8 desktops, é heterogêneo. São equipamentos 
com hardware similar - mas não exatamente igual - mas com uma variedade de sistemas 
operacionais. Entre Windows e diferentes versões de Linux, temos atualmente 2 sistemas operacionais, sendo duas distribuições Linux (Fedora e CentOS) em 
versões variadas.\\

\textbf{Sub-atividades:}

Num primeiro momento e já contando com a recém aquisição do sistema NAS utilizado 
no cluster LTHN\ref{chap:cluster} a ideia inicial é ter contas individuais para os 
usuários\footnote{Hoje há uma conta única (\texttt{cfx}) por todos os usuários.} 
centralizadas em uma máquina. Com isso será possível logar individualmente em qualquer máquina Linux. Além disso, é desejável que os diretórios dos usuários 
(\texttt{/home/<user>}) estejam localizados no NAS com implementação de cotas e que 
esse diretório seja carregado localmente via NFS a cada \textit{login} de usuário.

\begin{itemize}
	\item[1] Implementação de contas centralizadas para usuários do LTHN.
	\item[2] Implementação de \texttt{/home} exportado via NFS para usuários.
	\item[3] Instalação de software de uso geral (Serpent, OpenFOAM, MCNP e outros) no disco compartilhado do NAS via NFS.
\end{itemize}

\textbf{Observações:}

As atividades aqui previstas não serão em nenhuma hipótese iniciadas antes da entrega do cluster do LTHN.\\

\textbf{\textit{Outcomes}}

A solução de contas locais permite controle de uso de espaço em disco, backup 
centralizado, evita duplicidade de dados e dificulta, ainda, confusão de versões 
de arquivos.\\

\textbf{Dificuldades/Restrições}

A solução proposta exige administração local de contas (em outras palavras: tempo 
de gerenciamento de recursos computacionais) e não resolve o problema crônico 
de acesso às impressoras do CDTN\footnote{Este problema já foi levado e é de conhecimento do SETIN. Uma eventual solução para contas corporativas está sendo 
investigada por eles.}

A eventual solução corporativa de contas evita a gerência local e muito provavelmente permitirá o uso de impressoras. Entretanto, \textbf{é necessário 
que as máquinas servidoras sejam capazes de se comunicar via MPI} com o objetivo 
de simulações-teste antes que o problema completo seja submetido ao cluster. A 
solução corporativa não contempla este cenário.\\

\textbf{Última revisão:} \today
