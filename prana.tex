\chapter{PyPrana}

\textbf{Motivação:} $\star\star\star$\\

\textbf{Tipo:} Projeto.\\

\textbf{Descrição:} O objetivo deste projeto é re-escrever um sub-conjunto do 
software prana que compreenda inicialmente apenas os algorimos e funções 
utilizados pelo André no seu cálculo de PIV. Não está no escopo do projeto 
interface gráfica como o original do Prana.

A justificativa para esse projeto é de ordem prática e didática. A licença do 
MATLAB tem alto custo (e grande dependência de toolboxes) e uma implementação 
aberta nos permitiria interromper essa dependência. Do ponto de vista didático, 
essa implementação nos permitiria tentar fazer uso das aceleradoras gráficas 
do cluster com o desenvolvimento de software específico para aumento de 
desempenho na análise de dados.

A validação dos resultados pode ser feita com o Prana original executado no 
cluster.

\textbf{Sub-atividades}

\begin{itemize}
	\item[1] Descrição do algoritmo utilizado para análise do PIV do André 
	formalmente (texto) e visualmente, objetivando futura documentação do 
	código produzido;
	\item[2] Implementação em Python das funções de leitura de imagens.
\end{itemize}

\textbf{Observações:}

A tentativa de utilização das aceleradoras gráficas pode não levar a resultados 
melhores do que os de uma paralelização em CPUs.

A utilização das GPUs só é possível após domínio do algoritmo sequencial e 
eventual paralelização em CPUs.

\textbf{\textit{Outcomes}}

\begin{itemize}
	\item[1] Paper na revista FOSS caso o software alcance maturidade para tal. Caso contrário, publicação no INAC 2019.
	\item[2] Paper em revista com análise dos resultados do André na TAMU.

\end{itemize}


\textbf{Dificuldades/Restrições}

Dificuldade inerente da compreensão e implementação de algoritmos dessa complexidade.\\

\textbf{Revisões:}
\date{21 de novembro de 2018}
\date{30 de novembro de 2018}

%\date{\today}
