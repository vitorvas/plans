\chapter{Implementaçao de cálculos usando GPU no Serpent}

\textbf{Motivação:} $\star\star\star\star\star$\\

\textbf{Tipo:} Projeto\\

\textbf{Descrição:} O objetivo deste projeto é o de fazer com que a atual 
implementação do Serpent2 (versão 2.1.30) seja capaz de fazer uso da capacidade de 
processamento de placas gráficas (GPU) para auxiliar nos cálculos de histórias de 
nêutrons.

\textbf{Sub-atividades:} (Em avaliação) \\

%\begin{itemize}
%	\item[1] 
%	\item[2]
%	\item[3]
%\end{itemize}

\textbf{Observações:}

Este software será construído utilizando a biblioteca \textit{VTK}. Esta biblioteca 
é originalmente escrita em C e C++, mas possui \textit{bindings} para Python. Ainda não está decidida qual linguagem será utilizada na sua construção. Protótipos 
iniciais estão sendo feitos em Python.\\

\textbf{\textit{Outcomes}}

\begin{itemize}
	\item[1] Um software completo e registrável como produto tecnológico no 
INPI;
	\item[2] Publicação \textit{Journal of Open Source Software};
	\item[3] Publicação \textit{Annals of Nuclear Engineering} com exemplos de 
	utilização.
\end{itemize}

\textbf{Dificuldades/Restrições}

Para a publicação no JOSS, não basta o software com código aberto, mas é necessária 
documentação formal e em formato aberto, bem como testes de software também formais.\\

\textbf{Revisões:}
\date{21/11/2018}
%\date{\today}
