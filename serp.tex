\chapter{Implementaçao de cálculos usando GPU no Serpent2}

\textbf{Motivação:} $\star\star\star\star\star$\\

\textbf{Tipo:} Projeto\\

\textbf{Descrição:} O objetivo deste projeto é o de fazer com que a atual 
implementação do Serpent2 (versão 2.1.30) seja capaz de fazer uso da capacidade de 
processamento de placas gráficas (GPU) para auxiliar nos cálculos de histórias de 
nêutrons.

\textbf{Sub-atividades:} (Em avaliação) \\

\begin{itemize}
	\item[1] Compreensão da forma como são feitos os cálculos de nêutrons na atual 
	versão do Serpent, primeiramente sequencialmente, depois em paralelo;
	\item[2] Descrição do(s) algoritmo(s) e gerenciamento de memória;
	\item[3] Implementação de protótipo utilizando OpenCL.
\end{itemize}

\textbf{Observações:}

Este software será uma modificação em software já existente cuja licença permite 
modificações para uso pessoal. Caso cheguemos a uma implementação viável, deverão 
ser estudadas formas de cessão do software para os autores originais.\\

\textbf{\textit{Outcomes}}

\begin{itemize}
	\item[1] Uma biblioteca específica registrável como produto tecnológico no 
INPI (sob condições já que é baseada num software restrito);
	\item[2] Publicação \textit{Annals of Nuclear Engineering} com exemplos de 
	utilização e comparação com versão sem GPU;
	\item[3] O conhecimento no uso de GPUs em computação científica é um diferencial, o que poderá trazer colaborações externas internacionais no âmbito de um posdoc;
	\item[4] O \textit{know-how} pode ser aplicável a outros projetos \ref{pyrana, pytera} do próprio LTHN que podem tirar proveito do novo cluster.
\end{itemize}

\textbf{Dificuldades/Restrições}

Pode ser impossível alcançar uma implementação em GPU mais eficiente do que 
a atual versão do Serpent2. Não está clara a forma de colaboração/cessão do software caso seja implementada uma versão mais rápida.\\
