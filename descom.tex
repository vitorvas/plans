\chapter{Descomissionamento do TRIGA}

\textbf{Motivação:} $\star$\\

\textbf{Tipo:} Projeto (que eventualmente virará  uma atividade)\\

\textbf{Descrição:} Dentre os requisitos para o completo licenciamento do reator TRIGA apontados pela DRS, está a necessidade de um plano de descomissionamento para a instalação. Este plano existe mas necessitará de passar por revisão. 

O objetivo deste projeto é atender \textbf{todas} as demandas relativas ao descomissionamento do reator TRIGA num capítulo de acordo com a divisão de tarefas 
feita pela Graiciany e aprovada pela chefia.

\textbf{Sub-atividades:}

Esta é apenas uma descrição primária e sem avaliação consistente das eventuais 
sub-atividades a serem realizadas por mim neste projeto.

\begin{itemize}
	\item[1] Revisão do atual plano de descomissionamento do reator TRIGA;
	\item[2] Revisão bibliográfica a partir de \textit{technical reports} da AIEA 
	sobre o conteúdo esperado de um plano de descomisisonamento para reatores de pesquisa;
	\item[3] Atendimento ao definido pelas normas CNEN, em especial NE 1.04, NN 2.02 e normas do grupo 9 (avaliação preliminar).
\end{itemize}

\textbf{Observações:}\\

Cabe lembrar que há no CDTN duas pessoas historicamente envolvidas com atividades relativas ao descomissionamento do TRIGA: Pablo e Clédola Cássia.\\

\textbf{\textit{Outcomes}}

Capítulo relativo ao descomissionamento para o licenciamento do TRIGA, eventual 
\textit{paper} sobre as particularidades do TRIGA IPR-R1 relacionadas ao descomissionamento.\\

\textbf{Dificuldades/Restrições}
Podem haver dificuldades políticas na execução desta tarefa já que os responsáveis 
históricos por tratar do tema podem não estar abertos ao meu envolvimento. Podem ocorrer, ainda, dificuldades de ordem prática, já que o Pablo - responsável pelo capítulo - é também chefe da Radioproteção e substituto do chefe de divisão, o que 
o limita na disponibilidade de horários.

Apesar da experiência dos servidores historicamente envolvidos com o descomissionamento, não se sabe a motivação de ambos para tratar deste tema.


\textbf{Revisões:}
\date{17 de dezembro de 2018}
