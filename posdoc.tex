\chapter{Prospecção de POSDOC 2012}
\label{posdoc}

\textbf{Motivação:} $\star\star\star\star\star$\\

\textbf{Tipo:} Projeto\\

\textbf{Descrição:} Para propor, nos termos dos eventuais editais de pós-doutorado no exterior CAPES/CNPq, um projeto factível, executável e competitivo, é necessário 
desde já conectar as necessidades do projetos CDTN/CNEN com meus interesses técnicos juntamente com uma proposta representativa para a instituição anfitriã 
no exterior.

Tenho hoje dois caminhos a pensar para o posdoc:

\section{Proposição TRIGA}
 Trabalho em reatores do tipo TRIGA com vistas ao eventual descomissionamento do nosso reator TRIGA IPR-R1 juntamete com atividades de desenvolvimento computacional. Esta talvez seja uma proposta de difícil de planejar no tocante 
 ao casamento das atividades. Uma atividade de descomissionamento pode ter pouco 
 a se relacionar com uma atividade de desenvolvimento.
 
 Possíveis centros com reatores TRIGA:
 \begin{enumerate}
 	\item Universidade de Pavia, IT. Tem TRIGA, seu pouco das atividades deles.
 	\item Kansas State Univesisty: Possuem um TRIGA e o prof. James publicou sobre acoplamento na mesma época que eu. É uma opção promissora para unir TRIGA/desenvolvimento. Entretanto, não é um deparamento de Eng. Nuclear importante nos EUA.
 	\item Slovenia: os contatos existem com o Daniel. TRIGA e muito trabalho em Eng. Nuclear. Além do mais, o CDTN tem um convênio assinado com eles, o que pode 
 	ser um facilitador ao submeter o projeto ao órgão de fomento.
 \end{enumerate}

\section{Proposição Computação científica}

Nesta proposta, o objetivo é desenvolver alguma \textit{feature} necessária a um 
dos softwares utilizados por nós ou voltar a atacar o problema acoplado, que 
continua de interesse da comunidade.

As dificuldades, aqui, estão numa justificativa robusta de modo a convencer o 
órgão de fomento que vale a pena apoiar tal proposta que não proponha experimentos 
no exterior.

Possíveis centros para Computação Científica:
\begin{enumerate}
	\item VTT, Finlândia. São os desenvolvedores do código de Monte Carlo, que utilizamos no LTHN e temos o código-fonte. Conheço pessoalmente parte da equipe e, 
	com uma proposta relativamente simples de desenvolvimento, seria fácil ser aceito por eles. O TRIGA da VTT foi descomissionado em 2016.
	\item NCSU, North Carolina State University, EUA. A Maria Avramova é manda-chuva aqui e o André a conhece pessoalmente. Me passou o e-mail. Andou trabalhando com Monte Carlo e se supõe que não seja difícil conseguir a assinatura dela pra autorizar minha ida pra lá acoplar códigos. É necessário iniciar os contatos. É uma universidade respeitada na área nuclear.
	\item Wisc, Madison, EUA. O departamento é relativamente conhecido e não conheço ninguém lá. Entretanto, é um dos pioneiros no desenvolvimento de software na área nuclear de forma colaborativa. Isso permite iniciar uma colaboração independente e depois criar conexões com os pesquisadores. Eles desenvolvem um simulador Monte Carlo e querem adaptá-lo ao Serpente. Isso só já seria um pós-doc. Olhar uma eventual implementação de Delta-Tracking no código deles. Referências: \url{http://cnerg.github.io/projects/main.html} e \url{http://svalinn.github.io/DAGMC/}.
	
\end{enumerate}
\textbf{Observações:}



\textbf{\textit{Outcomes}}

Realização de 1 ano de pós-doutorado no exterior. Eventual (e esperada) publicação 
de trabalhos científicos. Reforço ou início de cooperação internacional.

\textbf{Dificuldades/Restrições}

Definir um projeto consistente e que desperte o interesse da CAPES/CNPq ao mesmo 
tempo em que se justifique a estadia no exterior.


