\chapter{Prospecção de POSDOC 2012}
\label{posdoc}

\textbf{Motivação:} $\star\star\star\star\star$\\

\textbf{Tipo:} Projeto\\

\textbf{Descrição:} Para propor, nos termos dos eventuais editais de pós-doutorado no exterior CAPES/CNPq, um projeto factível, executável e competitivo, é necessário 
desde já conectar as necessidades do projetos CDTN/CNEN com meus interesses técnicos juntamente com uma proposta representativa para a instituição anfitriã 
no exterior.

Tenho hoje dois caminhos a pensar para o posdoc:

\section{Proposição TRIGA}
% Trabalho em reatores do tipo TRIGA com vistas ao eventual descomissionamento do nosso reator TRIGA IPR-R1 juntamente com atividades de desenvolvimento computacional. Esta talvez seja uma proposta de difícil planejamento no tocante ao casamento das atividades. Uma atividade de descomissionamento pode ter pouco a se relacionar com uma atividade de desenvolvimento.
 
A ideia inicial de envolver descomissionamento foi abortada. Segue abaixo a lista inicial prospectiva de possíveis centros com reatores TRIGA:
 \begin{enumerate}
 	\item Universidade de Pavia, IT. Tem TRIGA, seu pouco das atividades deles.
 	\item Kansas State Univesity: Possuem um TRIGA e o prof. James publicou sobre acoplamento na mesma época que eu. É uma opção promissora para unir TRIGA/desenvolvimento. Entretanto, não é um deparamento de Eng. Nuclear importante nos EUA.
 	\item Slovenia: os contatos existem com o Daniel. TRIGA e muito trabalho em Eng. Nuclear. Além do mais, o CDTN tem um convênio assinado com eles, o que pode 
 	ser um facilitador ao submeter o projeto ao órgão de fomento.
 \end{enumerate}

\subsection{Kansas State University - KSU}

Acompanhando as atividades de pesquisa do professor Jeremy Roberts, foi publicado 
(aparentemente em 2018, via Research Gate) um trabalho sobre processos de predição de composição de combustíveis de reatores nucleares (\textit{Comparison Between Gaussian Processes and DMD Surrogates for Isotopic Composition Predicition}\cite{Roberts}).

O trabalho citado tem como objetivo estimar as atuais composições isotópicas 
do reator KSU TRIGA Mark II. Essa metodologia visa reduzir o tempo gasto em 
simulações e também visa atacar o problema de que eles não possuem o histórico 
detalhado de queima dos combustíveis avaliados. \textbf{Essa não é a mesma 
situação do CDTN}, cabe enfatizar. O combustível TRIGA é discretizado em 
três zonas axiais.

Apesar das diferenças entre KSU e CDTN, este trabalho pode servir de base 
para um projeto de pos-doc que seja também de interesse da KSU.

Atividades?
\begin{itemize}
\item Avaliar o momento de contactar o prof. Roberts;
\item Acrescentar neste documento os trabalhos similares aos meus.
\end{itemize}



\section{Proposição Computação científica}

Nesta proposta, o objetivo é desenvolver alguma \textit{feature} necessária a um 
dos softwares utilizados por nós ou voltar a atacar o problema acoplado, que 
continua de interesse da comunidade.

As dificuldades, aqui, estão numa justificativa robusta de modo a convencer o 
órgão de fomento que vale a pena apoiar tal proposta que não proponha experimentos 
no exterior.

Possíveis centros para Computação Científica:
\begin{enumerate}
	\item VTT, Finlândia. São os desenvolvedores do código de Monte Carlo, que utilizamos no LTHN e temos o código-fonte. Conheço pessoalmente parte da equipe e, 
	com uma proposta relativamente simples de desenvolvimento, seria fácil ser aceito por eles. O TRIGA da VTT foi descomissionado em 2016.
	\item NCSU, North Carolina State University, EUA. A Maria Avramova é manda-chuva aqui e o André a conhece pessoalmente. Me passou o e-mail. Andou trabalhando com Monte Carlo e se supõe que não seja difícil conseguir a assinatura dela pra autorizar minha ida pra lá acoplar códigos. É necessário iniciar os contatos. É uma universidade respeitada na área nuclear.
	\item Wisc, Madison, EUA. O departamento é relativamente conhecido e não conheço ninguém lá. Entretanto, é um dos pioneiros no desenvolvimento de software na área nuclear de forma colaborativa. Isso permite iniciar uma colaboração independente e depois criar conexões com os pesquisadores. Eles desenvolvem um simulador Monte Carlo e querem adaptá-lo ao Serpente. Isso só já seria um pós-doc. Olhar uma eventual implementação de Delta-Tracking no código deles. Referências: \url{http://cnerg.github.io/projects/main.html} e \url{http://svalinn.github.io/DAGMC/}. Eles publicam vagas sempre relacionadas a desenvolvimento. O contato, via site, é Paul Wilson (paul.wilson@wisc.edu), Professor, e líder do CNERG.
	
\end{enumerate}

\subsection{CNERG - WISC}
	Do ponto de vista de grupo de alto-nível, este é sem dúvida um dos principais. Realmente o trabalho é devotado a desenvolvimento. Acho que vale a tentativa de contato (a partir de julho de 2019).
\\


\textbf{Observações:}
\\


\textbf{\textit{Outcomes}}

Realização de 1 ano de pós-doutorado no exterior. Eventual (e esperada) publicação 
de trabalhos científicos. Reforço ou início de cooperação internacional.

\textbf{Dificuldades/Restrições}

Definir um projeto consistente e que desperte o interesse da CAPES/CNPq ao mesmo 
tempo em que se justifique a estadia no exterior.


