\chapter{Novo cluster: instalação e gerência}

\textbf{Motivação:} $\star\star\star\star$\\

\textbf{Tipo:} Projeto que se tornará atividade.\\

\textbf{Descrição:} Este projeto trata da instalação do novo cluster do Laboratório 
de Termohidráulica e Neutrônica do CDTN (LabTHN). Este sistema compreende uma máquina mestra e 8 máquinas escravas, todas equipadas com placas gráficas Nvidia que podem ser utilizadas tanto para visualização quanto para cálculos em paralelo.

O sistema está parcialmente instalado com Linux CentOS 7.5 com um serviço de 
sistema de arquivos distribuído Gluster ativo. 

O objetivo final deste projeto é ter o cluster funcionando com usuários com cotas 
de disco e com os seguintes sistemas instalados:

\begin{itemize}
	\item Serpent Monte Carlo;
	\item \textbf{MCNP 6 [INSTALADO]};
	\item OpenFOAM versão 6;
	\item \textbf{Pacote ANSYS [INSTALADO (cfx e fluent testados)]};
	\item Scale;
	\item \textbf{Matlab [INSTALADO (aguardando testes do Prana]}.
\end{itemize}

No presente momento, o sistema NAS está instalado em modo padrão, com duas 
interfaces de rede 10Gbits conectadas as switch do cluster (endereços IP 
temporários: \texttt{13.13.13.51} e \texttt{13.13.13.52}) e uma interface 
1Gb conectada à rede do CDTN. A arquitetura a ser usada no NAS ainda está sendo 
estudada, mas já foi configurado um volume em RAID 10 para o backup das máquinas 
do LabTHN.

\textbf{Sub-atividades}

A partir da situação atual do cluster, as sequintes atividades estão previstas:
\begin{itemize}
	\item[1] \textbf{Modificação da forma de autenticação dos hosts de chaves 
	para autenticação no host}. Isso é fundamental para simplificar a criação 
	de contas de usuários e diminuir o volume de dados de chaves armazenadas em 
	um diretório \texttt{/home} compartilhado;
	\item[2] Instalação e testes dos softwares descritos;
	\item[3] Instalação do TORQUE para lançamento de aplicações e controle de 
	carga de trabalho do sistema;
	\item[4] Implementação de capacidade de visualização remota pelas placas gráficas a partir de outras máquinas Linux do LabTHN.
	\item[5] Avaliação da viabilidade de um hub para exploração dos sistemas de segurança do cluster (power supplies e no-break).
	\item[6] Avaliação de implantação de desligamento automático em casa de falta de energia atráves de conexão lógica com o sistema de controle do no-break.
\end{itemize}

As sub-atividades listadas são atualizadas a cada revisão no documento, sendo que 
as sub-atividades finalizadas são removidas da lista e, de acordo com a demanda, 
são acrescidas novas tarefas. Por exemplo, a instalação do NAS ou a instalação do 
matlab, que não estavam no projeto inicial do cluster.

\textbf{Observações:}

O processo de instalação pode levar a demandas não previstas, de acordo com as avalições feitas para implantação de determinada \textit{feature} no sistema.

Uma vez instalado e testado, o projeto passa a ser uma atividade. Esta atividade 
consiste na manutenção e verificação do funcionamento do sistema.\\

\textbf{\textit{Outcomes}}

\begin{itemize}
	\item[1] Sistema de cálculos para as atividades do LabTHN e, eventualmente, para usuários externos.
	\item[2] Paper INAC 2019 com descrição do processo final de instalação e exemplos de aplicações com tempo de execução.

\end{itemize}


\textbf{Dificuldades/Restrições}

O sistema está hoje fora da rede elétrica do gerador.\\

\textbf{Revisões:}\\
\date{21 de novembro de 2018}\\
\date{30 de novembro de 2018}\\
\date{07 de dezembro de 2018}\\
\date{\today}

%\date{\today}
