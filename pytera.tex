\chapter{Desenvolvimento do \textit{Pytera}}
\label{pytera}

\textbf{Motivação:} $\star\star$\\

\textbf{Tipo:} Projeto\\

\textbf{Descrição:} O Pytera pretende ser um código para cálculo de subcanais 
nos moldes do Pantera, mas escrito em Python já desenvolvido com capacidade 
de execução em paralelo. Num segundo momento, pretende-se estender o paralelismo 
para tirar proveito das placas gŕaficas (GPUs) presentes no cluster do Laboratório 
de Termo-hidráulica e Neutrônica (LTHN).

Este deve ser o trabalho de doutorado do Diego da INB, sendo que minha participação 
será na orientação do desenvolvimento de acordo com as práticas de documentação, 
testes e desenvolvimento nos moldes de software livre.

A validação do desenvolvimento deverá ser feita com o Pantera original. Para isso, deverão 
ser executadas algumas tarefas básicas do ponto de vista de desenvolvimento de software.

\textbf{Sub-atividades:}

\begin{itemize}
	\item[1] Compilação do Pantera no ambiente de desenvolvimento a ser desenvolvido o Pytera. 
	Inicialmente, está pensado unicamente para Linux. A compilação em ambiente utilizado atualmente 
	é fundamental pois, para validar a nova versão, é preciso ter a versão anterior funcionando.
	\item[2] Devem ser feitos \textit{regression tests}. Estes visam garantir que as funcionalidades 
	implementadas geram os mesmos resultados que o programa original. Cabe ressaltar que, no caso de \textit{bugs}, 
	estes devem ser corrigidos.
%	\item[3]
\end{itemize}

\textbf{Observações:}

Pode não valer a pena a utilização de GPUs para o problema em questão. Além disso, 
como o desenvolvimento se dará em Python, é necessário buscar bibliotecas que 
ofereçam (numba, por exemplo) a possibilidade de usar as GPUS do Python.\\

O procedimento relativo a \textit{regression tests} deve ser igualmente realizado para 
o projeto PyPrana.\\

Uma boa referência para esta atividade é o artigo: "Heirloom Software: the Past as Adventure", de Eric S. Raymond, 
publicado na Linux Journal, setembro de 2017, páginas 108-118.

\textbf{\textit{Outcomes}}

Software com potencial para registro no INPI, potencial de publicação na JOSS e 
potencial de publicação em revista da área Nuclear/Termo-hidráulica (Annals of 
Nuclear Engineering ou Nuclear Engineering and Design).

Obviamente, o resulado principal esperado é o doutorado do aluno em questão.

\textbf{Dificuldades/Restrições}

Ainda não analisadas, mas provavelmente serão as dificuldades do desenvolvimento 
propriamente dito.

\textbf{Revisões:}

\date{\today}
