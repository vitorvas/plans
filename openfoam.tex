\chapter{\textcolor{red}{FINALIZADO}: OpenFOAM solver} 
%\textbf{Motivação:} $\star$\\

\textbf{Tipo:} Projeto - \texttt{sourceTermChtMultiRegionFoam}\\

\textbf{Descrição:}

Este projeto visa modificar o solver \texttt{chtMultiRegionFoam} do 
OpenFOAM versão 6 para utilizar internamente um termo-fonte nas equações 
de energia.

Apesar de ser possível (e já termos feito) uso de termo-fonte através do 
metódo \textit{fvOptions} do OpenFOAM, não conseguimos descrever um termo-fonte
não uniforme, ou seja, não foi possível passar como termo-fonte um campo não-uniforme
de densidade de potências. Além disso, com o termo-fonte interno é mais facilmente 
visualizável o resultado pelo paraview tanto na sua forma original como interpolada, esta
última uma \textit{feature} do próprio paraview.

O objetivo desse projeto é ter um solver atualizado para ser usado no acoplamento 
OpenFOAM+Serpent a ser feito como tema de dissertação de mestrado 
do Tiago.\\

\textbf{Sub-atividades:}

\begin{itemize}
  \item[1] Contactar o NIT/CDTN para viabilizar o registro do software.
\end{itemize}

\textbf{Observações:}

O solver está em:\\
 \url{https://gitlab.com/vitorvas/sourcetermchtmultiregionfoam.git}\\

\textbf{\textit{Outcomes}}

\begin{itemize}
	\item Programa de computador registrado (produção tecnológica de inovação);
	\item Paper a ser submetido para o INAC 2019.
\end{itemize}

%\textbf{Dificuldades/Restrições}

%\textbf{Última revisão:} \today
