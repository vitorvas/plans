\chapter{OpenFOAM solver \texttt{chtMultiRegionFoam}}

\textbf{Motivação:} $\star$\\

\textbf{Tipo:} Projeto\\

\textbf{Descrição:}

Este projeto visa modificar o solver \texttt{chtMultiRegionFoam} do 
OpenFOAM versão 6 para utilizar internamente um termo-fonte nas equações 
de energia.

Apesar de ser possível (e já termos feito) uso de termo-fonte através do 
metódo \textit{fvOptions} do OpenFOAM, acreditamos ser mais interessante 
modificar o solver. São modificações que já foram feitas anteriormente 
para a versão 2.2. Além disso, com o termo-fonte interno é mais facilmente 
visualizável o resultado pelo paraview.

O objetivo desse projeto é ter um solver atualizado para ser usado no acoplamento 
OpenFOAM+Serpent a ser (provavelmente) feito como tema de dissertação de mestrado 
do Tiago.\\

\textbf{Sub-atividades:}

\begin{itemize}
	\item[1] Modificar o arquivo \texttt{createSolidFields.H} para criar o volScalar para armazenar o termo-fonte; 
	\item[2] Modificar o arquivo \texttt{solveSolid.H} para incluir o termo-fonte na equação de energia;
	\item[3] Verificar se são necessárias modificações no arquivo \texttt{createSolidMeshes.H} para leitura do campo scalar para o termo-fonte.
\end{itemize}

\textbf{Observações:}

A modificação deve ser feita a partir da criação em separado de um solver 
(\texttt{chtMultiRegionSourceFoam}) pela cópia dos arquivos em um diretório qualquer.

Os arquivos de compilação utilizados pelo wmake devem refletir essas modificações.

É importante, para além da modificação do solver, ser capaz de fazer convergir os 
casos de testes originais do OpenFOAM e o caso adaptado da minha tese (disponível 
no github em: https://github.com/vitorvas/teste5)

Na máquina caprara-lx: \$FOAM\_APP/solvers/heatTransfer/chtMultiRegionFoam\\

\textbf{\textit{Outcomes}}

Esta modificação é muito simples para gerar um programa registrável. Entretanto, 
num cenário com scripts para a execução acoplada da dissertação do Tiago, isso 
pode ser viável.\\

\textbf{Dificuldades/Restrições}

Não se espera nada além das mesmas dificuldades da criação do solver para minha 
tese. \\

%\textbf{Última revisão:} \today
