\chapter{Planejamento e construção de um sistema de neutrongrafia digital para o TRIGA}

\textbf{Motivação:} $\star\star\star\star\star$\\

\textbf{Tipo:} Projeto (que eventualmente virará  uma atividade)\\

\textbf{Descrição:} O objetivo deste projeto é verificar a possibilidade e aplicabilidade de um sistema
de neutrongrafia digital no reator TRIGA e, eventualmente, construir tal sistema com equipamentos
de baixo custo e com referência na documentação da AIEA no tema, bem como, colocar em prática
o aprendido durante minha participação no AUNIRA 2019 (Daejeon, Coréia do Sul).

\textbf{Sub-atividades:}

Esta é apenas uma descrição primária e sem avaliação consistente das eventuais 
sub-atividades a serem realizadas por mim neste projeto.

\begin{itemize}
\item[1] Revisão bibliográfica a partir de \textit{technical reports} da AIEA, da documentação
  disponibilizada durante o AUNIRA e nos papers fornecidos gentilmente pelo pessoal da KAERI e
  de Nuno Pessoa-Barradas da AIEA;
\item[2] Simulação do fluxo neutrônico utilizando os modelos do TRIGA já utilizados no LTHN com
  o acréscimo do extrator de nêutrons;
  \item[3] Simulação do feixe de nêutrons utilizando software específico \href{http://www.mcstas.org/}{McStas};
\end{itemize}

\textbf{Observações:}\\

Este projeto é um passo numa futura justificativa para treinamento como operador do TRIGA, sendo que será referência das minhas atividades lá.\\

\textbf{\textit{Outcomes}}\\

  As simulações podem vir a se tornar um paper, a depender dos resultados.\\

  Projeto de fomento para a compra de equipamentos.\\

\textbf{Dificuldades/Restrições}
Podem haver dificuldades políticas na execução desta tarefa já que a equipe do TRIGA e seus principais
usuários são pouco afeitos a mudanças ou mesmo diferente utilização da instalação.\\

Não é conhecida a motivação da equipe do reator para eventual participação neste projeto.


%\textbf{Última revisão:} \today
