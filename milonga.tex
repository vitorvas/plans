%\chapter{\textcolor{red}{FINALIZADO}: Paralelização do \textit{milonga}}
\chapter{Paralelização do \textit{feenoX}}

\textbf{Motivação:} $\star\star$\\

\textbf{Tipo:} Projeto\\

\textbf{Descrição:}

\textbf{Em setembro/2021 o autor do milonga lançou o feenoX, que é a nova
  versão do milonga. Ainda tem TO-DOs relativos à paralelização como outros.
  Segue como projeto de interesse para 2023.}

\textbf{Uma das principais alterações é que não há mais wasora e há uma organização
  do código de modo a separar os modelos matemáticos (Laplace, etc).}

Este projeto objetiva a paralelização do código milonga. O milonga é um solver 
de elementos finitos e volumes finitos para a solução da equação de difusão de 
neutrons. Funciona sobre o código wasora, que oferece funções de base sobre as 
quais o milonga executa.

A ligação intrínseca entre ambos os softwares, torna complexa a paralelização. 
Uma primeira metodologia de paralelização consiste em separação de domínios. 
Em outras palavras, a malha é dividida entre processadores e nas faces separadas 
são definidas condições de contorno especiais nas quais a comunicação entre 
processadores pode ser feita (utilizando, por exemplo, MPI).

Uma formulação mais simples para a paralelização consiste na solução das matrizes 
já construídas, seja por elementos finitos ou volumes finitos, em paralelo. Em 
teoria isso é facilmente atingido uma vez que o milonga faz uso da biblioteca 
PETSc para essa solução e esta biblioteca é construída utilizando-se do MPI. 
Apesar de mais simples se comparada à proposta anterior, uma vez que são feitas 
chamadas espalhadas entre o milonga e o wasora de funções da PETSc para a construção das matrizes, uma paralelização nesse nível é não-trivial.

Em resumo, quaisquer opções são bastante trabalhosas em termos de investimento 
de tempo em re-codificação.

Cabe ressaltar que já foram feitos alguns estudos com profiling do milonga para 
diferentes casos e algumas funções que percorrem a malha são os \textit{bottlenecks} conhecidos. Modificações na malha para tratar deste problema 
são também complexas.\\

\textbf{Sub-atividades:}

\begin{itemize}
	\item[1] Separação da construção das matrizes com execução do código comum 
	apenas no processo mestre.
\end{itemize}

\textbf{Observações:}

Problema muito interessante do ponto de vista didático, para a compreensão 
de FEM e FVM, neutrônica e como são feitas as soluções. Entretanto, bastante 
complexa qualquer modificação no código.

O milonga e o wasora são ambos construídos unicamente usando C e são opensource.\\

\textbf{\textit{Outcomes}}

Código paralelizado (não sei se cabe registro), eventual paper sobre a parelização 
executando casos complexos numa revista como Annals, ganho de know-how para eventual futuro pós-doc e reconhecimento de uma pequena comunidade utilizadora 
do milonga.\\

\textbf{Dificuldades/Restrições}

Tempo e esforço envolvidos com risco alto de não ser viável a paralelização.\\

\textbf{Resultados}

Vários testes com exemploes do \textit{milonga} para malhas com diferentes números de elementos mostraram, para o método de difusão por volumes finitos, que grande parte 
do tempo de computação (cerca de $44\%$) é gasto em funções de percurso da malha.

Apenas uma máximo de $10\%$ do tempo é gasto em operações de funções do módulo 
de difusão. 

A finalização ideal deste projeto seria com um relatório formal de \textit{benchmark} de três diferentes tipos de malhas (em número de elementos) 
para cada tipo de formulação:

\begin{enumerate}
\item Difusão volumes finitos;
\item Difusão elementos finitos;
\item Ordenadas discretas ($S_N$) volumes finitos;
\item Ordenadas discretas ($S_N$) elementos finitos;
\end{enumerate}

Devido à minha saída de licença, esse relatório não seja realizado já que não há 
potencial de publicação. Entretanto, permanece o interesse didático, em especial no entendimento das formulações e no uso das bibliotecas PETSc e SLEPc.
