%Tudo que começa com '%' é comentário e é ignorado pelo compilador

%Gerando arquivo em latex:
%latex arquivo.tex (em dvi)
%pdflatex arquivo (em pdf)
%dvipdfm arquivo
%s2pdf arquivo

% Alguns modos de usar o latex:
% Windows – Miktex com Led
% Linux – texlive com kile

\documentclass[12pt]{report} %aqui fala o tipo de documento e o tamanho da fonte. Opções: tamanho do texto (10pt, 12pt, 14pt), formato do papel (a4paper, a5paper, b5paper, letterpaper, legalpaper, executivepaper), o número de colunas (onecolumn, twocolumn), entre outras opções.
%Por exemplo, [12pt,a4,twocolumn].
%classe: article, report, letter, book ou slides. Instalar abnt para quem está pensando no tf
\usepackage[brazilian]{babel} %hifenização em português do brasil
\usepackage[T1]{fontenc} % caracteres com acentos são considerados um bloco só
\usepackage[utf8]{inputenc} % Corrigie os acentos em Português
\usepackage{ae} %arruma a fonte quando usa o pacote fontenc
%\usepackage[pdftex]{graphicx}%Para inserir figuras
\usepackage{vhistory}

\begin{document}
\title{Projetos e atividades}
\author{Vitor Vasconcelos Araújo Silva\\ vitors@cdtn.br}
\date{\today}

\maketitle %cria o título

%\def \negritovi {\textbf} %Criando comandos

\tableofcontents %índice
\pagebreak % Quebra de página
%\listoffigures %indice de figuras
%\listoftables %indice de tabelas
\pagebreak % Quebra de página

\begin{abstract}
Este documento apresenta os projetos e atividades planejados e/ou execução 
dentro da minha área de atuação no SETRE/CDTN. Para cada um deles são 
brevemente apresentados objetivo, motivação, aplicações, possíveis \textit{outcomes} e nível de dificuldade de execução. O objetivo da 
condensação das atividades e projetos em um documento é o de simplificar 
a avaliação temporal da evolução dos planos e organizar o tempo investido 
em cada um deles.
\end{abstract}

\begin{versionhistory}
	\vhEntry{0.0}{21/11/2018}{Vitor}{Documento criado}
%	\vhEntry{1.1}{23.01.04}{DP|JPW}{correction}
%    \vhEntry{1.2}{03.02.04}{DP|JPW}{revised after review}
\end{versionhistory}

\chapter{Visualizador de geometrias Serpent}

\textbf{Motivação:} $\star\star\star\star\star$\\

\textbf{Tipo:} Projeto\\

\textbf{Descrição:} O objetivo deste projeto é de construir um software capaz 
de ler um arquivo de entrada do Serpent e gerar um modelo visual 3D da geomtria 
descrita. O Serpent utiliza primariamente métodos de geometria sólida construtiva 
(CSG) separada em universos para definir regiões para cálculo neutrônico. A 
contribuição esperada é permitir que os usuários do Serpent possam visualizar 
sua entrada em busca de partes irregulares ao, mesmo, visualizar partes internas 
ou elementos de interesse.

Não estão previstas inicialmente modificações na visualização como cortes em 
planos ou filtrons afins. A ideia é, uma vez gerados os elementos descritos no 
input, além de visualizá-los no monitor, gravar os elementos em um arquivo 
do formato VTK. Este arquivo, por sua vez, pode ser nativamente aberto pelo 
Paraview, que oferece diversos tipos de formas de manipulação dos dados para 
visualização.

Numa segunda etapa é interessante extender as classes relativas à leitura dos elementos Serpent para o formato MCNP, aumentando a aplicabilidade do software 
já que a base de usuários do MCNP é consideravelmente maior.\\

\textbf{Sub-atividades:} (Em avaliação) \\

%\begin{itemize}
%	\item[1] 
%	\item[2]
%	\item[3]
%\end{itemize}

\textbf{Observações:}

Este software será construído utilizando a biblioteca \textit{VTK}. Esta biblioteca 
é originalmente escrita em C e C++.\\%, mas possui \textit{bindings} para Python. Ainda não está decidida qual linguagem será utilizada na sua construção. Protótipos 
%iniciais estão sendo feitos em Python.\\
Durante o ano de licença (2020) codifiquei alguns protótipos simples usando C++. Foram experimentadas as classes a serem usadas e uma forma inicial de geração
dos sólidos. No protótipo final, um arquivo vtk é escrito a partir de geometria simples pré-definida. Resultados encorajadores.\\

No mesmo ano de 2020 a equipe da \href{https://vtk.org/}{Kitware} atualizou toda a documentação do \textit{VTK}.

\textbf{\textit{Outcomes}}

\begin{itemize}
	\item[1] Um software completo e registrável como produto tecnológico no 
INPI;
	\item[2] Publicação \textit{Journal of Open Source Software};
	\item[3] Publicação \textit{Annals of Nuclear Engineering} com exemplos de 
	utilização.
\end{itemize}

\textbf{Dificuldades/Restrições}

Para a publicação no JOSS, não basta o software com código aberto, mas é necessária 
documentação formal e em formato aberto, bem como testes de software também formais.\\

 %Cria uma seção
\chapter{Novo cluster: instalação e gerência}

\textbf{Motivação:} $\star\star\star\star$\\

\textbf{Tipo:} Projeto que se tornará atividade.\\

\textbf{Descrição:} Este projeto trata da instalação do novo cluster do Laboratório 
de Termohidráulica e Neutrônica do CDTN (LabTHN). Este sistema compreende uma máquina mestra e 8 máquinas escravas, todas equipadas com placas gráficas Nvidia que podem ser utilizadas tanto para visualização quanto para cálculos em paralelo.

O sistema está parcialmente instalado com Linux CentOS 7.5 com um serviço de 
sistema de arquivos distribuído Gluster ativo. 

O objetivo final deste projeto é ter o cluster funcionando com usuários com cotas 
de disco e com os seguintes sistemas instalados:

\begin{itemize}
	\item Serpent Monte Carlo;
	\item \textbf{MCNP 6 [INSTALADO]};
	\item OpenFOAM versão 6;
	\item \textbf{Pacote ANSYS [INSTALADO (cfx e fluent testados)]};
	\item Scale;
	\item \textbf{Matlab [INSTALADO (aguardando testes do Prana]}.
\end{itemize}

No presente momento, o sistema NAS está instalado em modo padrão, com duas 
interfaces de rede 10Gbits conectadas as switch do cluster (endereços IP 
temporários: \texttt{13.13.13.51} e \texttt{13.13.13.52}) e uma interface 
1Gb conectada à rede do CDTN. A arquitetura a ser usada no NAS ainda está sendo 
estudada, mas já foi configurado um volume em RAID 10 para o backup das máquinas 
do LabTHN.

\textbf{Sub-atividades}

A partir da situação atual do cluster, as sequintes atividades estão previstas:
\begin{itemize}
	\item[1] \textbf{Modificação da forma de autenticação dos hosts de chaves 
	para autenticação no host}. Isso é fundamental para simplificar a criação 
	de contas de usuários e diminuir o volume de dados de chaves armazenadas em 
	um diretório \texttt{/home} compartilhado;
	\item[2] Instalação e testes dos softwares descritos;
	\item[3] Instalação do TORQUE para lançamento de aplicações e controle de 
	carga de trabalho do sistema;
	\item[4] Implementação de capacidade de visualização remota pelas placas gráficas a partir de outras máquinas Linux do LabTHN.
	\item[5] Avaliação da viabilidade de um hub para exploração dos sistemas de segurança do cluster (power supplies e no-break).
	\item[6] Avaliação de implantação de desligamento automático em casa de falta de energia atráves de conexão lógica com o sistema de controle do no-break.
\end{itemize}

As sub-atividades listadas são atualizadas a cada revisão no documento, sendo que 
as sub-atividades finalizadas são removidas da lista e, de acordo com a demanda, 
são acrescidas novas tarefas. Por exemplo, a instalação do NAS ou a instalação do 
matlab, que não estavam no projeto inicial do cluster.

\textbf{Observações:}

O processo de instalação pode levar a demandas não previstas, de acordo com as avalições feitas para implantação de determinada \textit{feature} no sistema.

Uma vez instalado e testado, o projeto passa a ser uma atividade. Esta atividade 
consiste na manutenção e verificação do funcionamento do sistema.\\

\textbf{\textit{Outcomes}}

\begin{itemize}
	\item[1] Sistema de cálculos para as atividades do LabTHN e, eventualmente, para usuários externos.
	\item[2] Paper INAC 2019 com descrição do processo final de instalação e exemplos de aplicações com tempo de execução.

\end{itemize}


\textbf{Dificuldades/Restrições}

O sistema está hoje fora da rede elétrica do gerador.\\

\textbf{Revisões:}\\
\date{21 de novembro de 2018}\\
\date{30 de novembro de 2018}\\
\date{07 de dezembro de 2018}\\
\date{\today}

%\date{\today}


\end{document}