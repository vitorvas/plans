%Tudo que começa com '%' é comentário e é ignorado pelo compilador

%Gerando arquivo em latex:
%latex arquivo.tex (em dvi)
%pdflatex arquivo (em pdf)
%dvipdfm arquivo
%s2pdf arquivo

% Alguns modos de usar o latex:
% Windows – Miktex com Led
% Linux – texlive com kile

\documentclass[12pt]{report} %aqui fala o tipo de documento e o tamanho da fonte. Opções: tamanho do texto (10pt, 12pt, 14pt), formato do papel (a4paper, a5paper, b5paper, letterpaper, legalpaper, executivepaper), o número de colunas (onecolumn, twocolumn), entre outras opções.
%Por exemplo, [12pt,a4,twocolumn].
%classe: article, report, letter, book ou slides. Instalar abnt para quem está pensando no tf
\usepackage[brazilian]{babel} %hifenização em português do brasil
\usepackage[T1]{fontenc} % caracteres com acentos são considerados um bloco só
\usepackage[utf8]{inputenc} % Corrigie os acentos em Português
\usepackage{ae} %arruma a fonte quando usa o pacote fontenc
%\usepackage[pdftex]{graphicx}%Para inserir figuras
\usepackage{vhistory}
\usepackage{xr} % referenciar arquivos externos
\usepackage{hyperref} % Pra link http funcionar

\externaldocument{prana}
\externaldocument{pytera}


\begin{document}
\title{Projetos e atividades para 2019}
\author{Vitor Vasconcelos Araújo Silva\\ vitors@cdtn.br}
\date{\today}

\maketitle %cria o título

%\def \negritovi {\textbf} %Criando comandos

\tableofcontents %índice
%\pagebreak % Quebra de página
%\listoffigures %indice de figuras
%\listoftables %indice de tabelas
%\pagebreak % Quebra de página

\begin{abstract}
Este documento apresenta os projetos e atividades planejados e/ou execução 
dentro da minha área de atuação no SETRE/CDTN. Para cada um deles são 
brevemente apresentados objetivo, motivação, aplicações, possíveis \textit{outcomes} e nível de dificuldade de execução. O objetivo da 
condensação das atividades e projetos em um documento é o de simplificar 
a avaliação temporal da evolução dos planos e organizar o tempo investido 
em cada um deles.
\end{abstract}

%\begin{versionhistory}
%	\vhEntry{0.0}{21/11/2018}{Vitor}{Documento criado}
%	\vhEntry{0.1}{30/11/2018}{Vitor}{Correção e Atualização}
%	\vhEntry{0.2}{07/12/2018}{Vitor}{Correção e Atualização}
%	\vhEntry{0.3}{13/12/2018}{Vitor}{Correção e Atualização}
%%	\vhEntry{1.1}{23.01.04}{DP|JPW}{correction}
%%    \vhEntry{1.2}{03.02.04}{DP|JPW}{revised after review}
%\end{versionhistory}

\chapter{Visualizador de geometrias Serpent}

\textbf{Motivação:} $\star\star\star\star\star$\\

\textbf{Tipo:} Projeto\\

\textbf{Descrição:} O objetivo deste projeto é de construir um software capaz 
de ler um arquivo de entrada do Serpent e gerar um modelo visual 3D da geomtria 
descrita. O Serpent utiliza primariamente métodos de geometria sólida construtiva 
(CSG) separada em universos para definir regiões para cálculo neutrônico. A 
contribuição esperada é permitir que os usuários do Serpent possam visualizar 
sua entrada em busca de partes irregulares ao, mesmo, visualizar partes internas 
ou elementos de interesse.

Não estão previstas inicialmente modificações na visualização como cortes em 
planos ou filtrons afins. A ideia é, uma vez gerados os elementos descritos no 
input, além de visualizá-los no monitor, gravar os elementos em um arquivo 
do formato VTK. Este arquivo, por sua vez, pode ser nativamente aberto pelo 
Paraview, que oferece diversos tipos de formas de manipulação dos dados para 
visualização.

Numa segunda etapa é interessante extender as classes relativas à leitura dos elementos Serpent para o formato MCNP, aumentando a aplicabilidade do software 
já que a base de usuários do MCNP é consideravelmente maior.\\

\textbf{Sub-atividades:} (Em avaliação) \\

%\begin{itemize}
%	\item[1] 
%	\item[2]
%	\item[3]
%\end{itemize}

\textbf{Observações:}

Este software será construído utilizando a biblioteca \textit{VTK}. Esta biblioteca 
é originalmente escrita em C e C++.\\%, mas possui \textit{bindings} para Python. Ainda não está decidida qual linguagem será utilizada na sua construção. Protótipos 
%iniciais estão sendo feitos em Python.\\
Durante o ano de licença (2020) codifiquei alguns protótipos simples usando C++. Foram experimentadas as classes a serem usadas e uma forma inicial de geração
dos sólidos. No protótipo final, um arquivo vtk é escrito a partir de geometria simples pré-definida. Resultados encorajadores.\\

No mesmo ano de 2020 a equipe da \href{https://vtk.org/}{Kitware} atualizou toda a documentação do \textit{VTK}.

\textbf{\textit{Outcomes}}

\begin{itemize}
	\item[1] Um software completo e registrável como produto tecnológico no 
INPI;
	\item[2] Publicação \textit{Journal of Open Source Software};
	\item[3] Publicação \textit{Annals of Nuclear Engineering} com exemplos de 
	utilização.
\end{itemize}

\textbf{Dificuldades/Restrições}

Para a publicação no JOSS, não basta o software com código aberto, mas é necessária 
documentação formal e em formato aberto, bem como testes de software também formais.\\

 %Cria uma seção
\chapter{Novo cluster: instalação e gerência}

\textbf{Motivação:} $\star\star\star\star$\\

\textbf{Tipo:} Projeto que se tornará atividade.\\

\textbf{Descrição:} Este projeto trata da instalação do novo cluster do Laboratório 
de Termohidráulica e Neutrônica do CDTN (LabTHN). Este sistema compreende uma máquina mestra e 8 máquinas escravas, todas equipadas com placas gráficas Nvidia que podem ser utilizadas tanto para visualização quanto para cálculos em paralelo.

O sistema está parcialmente instalado com Linux CentOS 7.5 com um serviço de 
sistema de arquivos distribuído Gluster ativo. 

O objetivo final deste projeto é ter o cluster funcionando com usuários com cotas 
de disco e com os seguintes sistemas instalados:

\begin{itemize}
	\item Serpent Monte Carlo;
	\item \textbf{MCNP 6 [INSTALADO]};
	\item OpenFOAM versão 6;
	\item \textbf{Pacote ANSYS [INSTALADO (cfx e fluent testados)]};
	\item Scale;
	\item \textbf{Matlab [INSTALADO (aguardando testes do Prana]}.
\end{itemize}

No presente momento, o sistema NAS está instalado em modo padrão, com duas 
interfaces de rede 10Gbits conectadas as switch do cluster (endereços IP 
temporários: \texttt{13.13.13.51} e \texttt{13.13.13.52}) e uma interface 
1Gb conectada à rede do CDTN. A arquitetura a ser usada no NAS ainda está sendo 
estudada, mas já foi configurado um volume em RAID 10 para o backup das máquinas 
do LabTHN.

\textbf{Sub-atividades}

A partir da situação atual do cluster, as sequintes atividades estão previstas:
\begin{itemize}
	\item[1] \textbf{Modificação da forma de autenticação dos hosts de chaves 
	para autenticação no host}. Isso é fundamental para simplificar a criação 
	de contas de usuários e diminuir o volume de dados de chaves armazenadas em 
	um diretório \texttt{/home} compartilhado;
	\item[2] Instalação e testes dos softwares descritos;
	\item[3] Instalação do TORQUE para lançamento de aplicações e controle de 
	carga de trabalho do sistema;
	\item[4] Implementação de capacidade de visualização remota pelas placas gráficas a partir de outras máquinas Linux do LabTHN.
	\item[5] Avaliação da viabilidade de um hub para exploração dos sistemas de segurança do cluster (power supplies e no-break).
	\item[6] Avaliação de implantação de desligamento automático em casa de falta de energia atráves de conexão lógica com o sistema de controle do no-break.
\end{itemize}

As sub-atividades listadas são atualizadas a cada revisão no documento, sendo que 
as sub-atividades finalizadas são removidas da lista e, de acordo com a demanda, 
são acrescidas novas tarefas. Por exemplo, a instalação do NAS ou a instalação do 
matlab, que não estavam no projeto inicial do cluster.

\textbf{Observações:}

O processo de instalação pode levar a demandas não previstas, de acordo com as avalições feitas para implantação de determinada \textit{feature} no sistema.

Uma vez instalado e testado, o projeto passa a ser uma atividade. Esta atividade 
consiste na manutenção e verificação do funcionamento do sistema.\\

\textbf{\textit{Outcomes}}

\begin{itemize}
	\item[1] Sistema de cálculos para as atividades do LabTHN e, eventualmente, para usuários externos.
	\item[2] Paper INAC 2019 com descrição do processo final de instalação e exemplos de aplicações com tempo de execução.

\end{itemize}


\textbf{Dificuldades/Restrições}

O sistema está hoje fora da rede elétrica do gerador.\\

\textbf{Revisões:}\\
\date{21 de novembro de 2018}\\
\date{30 de novembro de 2018}\\
\date{07 de dezembro de 2018}\\
\date{\today}

%\date{\today}

\chapter{\textcolor{red}{FINALIZADO}: OpenFOAM solver} 
%\textbf{Motivação:} $\star$\\

\textbf{Tipo:} Projeto - \texttt{sourceTermChtMultiRegionFoam}\\

\textbf{Descrição:}

Este projeto visa modificar o solver \texttt{chtMultiRegionFoam} do 
OpenFOAM versão 6 para utilizar internamente um termo-fonte nas equações 
de energia.

Apesar de ser possível (e já termos feito) uso de termo-fonte através do 
metódo \textit{fvOptions} do OpenFOAM, não conseguimos descrever um termo-fonte
não uniforme, ou seja, não foi possível passar como termo-fonte um campo não-uniforme
de densidade de potências. Além disso, com o termo-fonte interno é mais facilmente 
visualizável o resultado pelo paraview tanto na sua forma original como interpolada, esta
última uma \textit{feature} do próprio paraview.

O objetivo desse projeto é ter um solver atualizado para ser usado no acoplamento 
OpenFOAM+Serpent a ser feito como tema de dissertação de mestrado 
do Tiago.\\

\textbf{Sub-atividades:}

\begin{itemize}
  \item[1] Contactar o NIT/CDTN para viabilizar o registro do software.
\end{itemize}

\textbf{Observações:}

O solver está em:\\
 \url{https://gitlab.com/vitorvas/sourcetermchtmultiregionfoam.git}\\

\begin{verbatim}
[cfx@caprara-lx sourceTermChtMultiRegionFoam]$ tree

8 directories, 43 files
\end{verbatim}

\textbf{\textit{Outcomes}}

\begin{itemize}
	\item Programa de computador registrado (produção tecnológica de inovação) (\textbf{Solicitado});
	\item Paper a ser submetido para o INAC 2019. (\textbf{Não gerou paper})
\end{itemize}

%\textbf{Dificuldades/Restrições}

%\textbf{Última revisão:} \today

\chapter{Paralelização do \textit{milonga}}

\textbf{Motivação:} $\star\star$\\

\textbf{Tipo:} Projeto\\

\textbf{Descrição:}

Este projeto objetiva a paralelização do código milonga. O milonga é um solver 
de elementos finitos e volumes finitos para a solução da equação de difusão de 
neutrons. Funciona sobre o código wasora, que oferece funções de base sobre as 
quais o milonga executa.

A ligação intrínseca entre ambos os softwares, torna complexa a paralelização. 
Uma primeira metodologia de paralelização consiste em separação de domínios. 
Em outras palavras, a malha é dividida entre processadores e nas faces separadas 
são definidas condições de contorno especiais nas quais a comunicação entre 
processadores pode ser feita (utilizando, por exemplo, MPI).

Uma formulação mais simples para a paralelização consiste na solução das matrizes 
já construídas, seja por elementos finitos ou volumes finitos, em paralelo. Em 
teoria isso é facilmente atingido uma vez que o milonga faz uso da biblioteca 
PETSc para essa solução e esta biblioteca é construída utilizando-se do MPI. 
Apesar de mais simples se comparada à proposta anterior, uma vez que são feitas 
chamadas espalhadas entre o milonga e o wasora de funções da PETSc para a construção das matrizes, uma paralelização nesse nível é não-trivial.

Em resumo, quaisquer opções são bastante trabalhosas em termos de investimento 
de tempo em re-codificação.

Cabe ressaltar que já foram feitos alguns estudos com profiling do milonga para 
diferentes casos e algumas funções que percorrem a malha são os \textit{bottlenecks} conhecidos. Modificações na malha para tratar deste problema 
são também complexas.\\

\textbf{Sub-atividades:}

\begin{itemize}
	\item[1] Separação da construção das matrizes com execução do código comum 
	apenas no processo mestre.
\end{itemize}

\textbf{Observações:}

Problema muito interessante do ponto de vista didático, para a compreensão 
de FEM e FVM, neutrônica e como são feitas as soluções. Entretanto, bastante 
complexa qualquer modificação no código.

O milonga e o wasora são ambos construídos unicamente usando C e são opensource.\\

\textbf{\textit{Outcomes}}

Código paralelizado (não sei se cabe registro), eventual paper sobre a parelização 
executando casos complexos numa revista como Annals, ganho de know-how para eventual futuro pós-doc e reconhecimento de uma pequena comunidade utilizadora 
do milonga.\\

\textbf{Dificuldades/Restrições}

Tempo e esforço envolvidos com risco alto de não ser viável a paralelização.

\chapter{PyPrana}

\textbf{Motivação:} $\star\star\star$\\

\textbf{Tipo:} Projeto.\\

\textbf{Descrição:} O objetivo deste projeto é re-escrever um sub-conjunto do 
software prana que compreenda inicialmente apenas os algorimos e funções 
utilizados pelo André no seu cálculo de PIV. Não está no escopo do projeto 
interface gráfica como o original do Prana.

A justificativa para esse projeto é de ordem prática e didática. A licença do 
MATLAB tem alto custo (e grande dependência de toolboxes) e uma implementação 
aberta nos permitiria interromper essa dependência. Do ponto de vista didático, 
essa implementação nos permitiria tentar fazer uso das aceleradoras gráficas 
do cluster com o desenvolvimento de software específico para aumento de 
desempenho na análise de dados.

A validação dos resultados pode ser feita com o Prana original executado no 
cluster.

\textbf{Sub-atividades}

\begin{itemize}
	\item[1] Descrição do algoritmo utilizado para análise do PIV do André 
	formalmente (texto) e visualmente, objetivando futura documentação do 
	código produzido;
	\item[2] Implementação em Python das funções de leitura de imagens.
\end{itemize}

\textbf{Observações:}

A tentativa de utilização das aceleradoras gráficas pode não levar a resultados 
melhores do que os de uma paralelização em CPUs.

A utilização das GPUs só é possível após domínio do algoritmo sequencial e 
eventual paralelização em CPUs.

\textbf{\textit{Outcomes}}

\begin{itemize}
	\item[1] Paper na revista FOSS caso o software alcance maturidade para tal. Caso contrário, publicação no INAC 2019.
	\item[2] Paper em revista com análise dos resultados do André na TAMU.

\end{itemize}


\textbf{Dificuldades/Restrições}

Dificuldade inerente da compreensão e implementação de algoritmos dessa complexidade.\\

\textbf{Revisões:}
\date{21 de novembro de 2018}
\date{30 de novembro de 2018}

%\date{\today}

\chapter{Desenvolvimento do \textit{Pytera}}

\textbf{Motivação:} $\star\star$\\

\textbf{Tipo:} Projeto\\

\textbf{Descrição:} O Pytera pretende ser um código para cálculo de subcanais 
nos moldes do Pantera, mas escrito em Python já desenvolvido com capacidade 
de execução em paralelo. Num segundo momento, pretende-se estender o paralelismo 
para tirar proveito das placas gŕaficas (GPUs) presentes no cluster do Laboratório 
de Termo-hidráulica e Neutrônica (LTHN).

Este deve ser o trabalho de doutorado do Diego da INB, sendo que minha participação 
será na orientação do desenvolvimento de acordo com as práticas de documentação, 
testes e desenvolvimento nos moldes de software livre.

A validação do desenvolvimento deverá ser feita com o Pantera original.

\textbf{Sub-atividades:}

\begin{itemize}
	\item \textbf{[A DEFINIR]}
%	\item[2]
%	\item[3]
\end{itemize}

\textbf{Observações:}

Pode não valer a pena a utilização de GPUs para o problema em questão. Além disso, 
como o desenvolvimento se dará em Python, é necessário buscar bibliotecas que 
ofereçam (numba, por exemplo) a possibilidade de usar as GPUS do Python.\\

\textbf{\textit{Outcomes}}

Software com potencial para registro no INPI, potencial de publicação na JOSS e 
potencial de publicação em revista da área Nuclear/Termo-hidráulica (Annals of 
Nuclear Engineering ou Nuclear Engineering and Design).

Obviamente, o resulado principal esperado é o doutorado do aluno em questão.

\textbf{Dificuldades/Restrições}

Ainda não analisadas, mas provavelmente serão as dificuldades do desenvolvimento 
propriamente dito.

\textbf{Revisões:}

\date{\today}

\chapter{Descomissionamento do TRIGA}

\textbf{Motivação:} $\star$\\

\textbf{Tipo:} Projeto (que eventualmente virará  uma atividade)\\

\textbf{Descrição:} Dentre os requisitos para o completo licenciamento do reator TRIGA apontados pela DRS, está a necessidade de um plano de descomissionamento para a instalação. Este plano existe mas necessitará de passar por revisão. 

O objetivo deste projeto é atender \textbf{todas} as demandas relativas ao descomissionamento do reator TRIGA num capítulo de acordo com a divisão de tarefas 
feita pela Graiciany e aprovada pela chefia.

\textbf{Sub-atividades:}

Esta é apenas uma descrição primária e sem avaliação consistente das eventuais 
sub-atividades a serem realizadas por mim neste projeto.

\begin{itemize}
	\item[1] Revisão do atual plano de descomissionamento do reator TRIGA;
	\item[2] Revisão bibliográfica a partir de \textit{technical reports} da AIEA 
	sobre o conteúdo esperado de um plano de descomisisonamento para reatores de pesquisa;
	\item[3] Atendimento ao definido pelas normas CNEN, em especial NE 1.04, NN 2.02 e normas do grupo 9 (avaliação preliminar).
\end{itemize}

\textbf{Observações:}\\

Cabe lembrar que há no CDTN duas pessoas historicamente envolvidas com atividades relativas ao descomissionamento do TRIGA: Pablo e Clédola Cássia.\\

\textbf{\textit{Outcomes}}

Capítulo relativo ao descomissionamento para o licenciamento do TRIGA, eventual 
\textit{paper} sobre as particularidades do TRIGA IPR-R1 relacionadas ao descomissionamento.\\

\textbf{Dificuldades/Restrições}
Podem haver dificuldades políticas na execução desta tarefa já que os responsáveis 
históricos por tratar do tema podem não estar abertos ao meu envolvimento. Podem ocorrer, ainda, dificuldades de ordem prática, já que o Pablo - responsável pelo capítulo - é também chefe da Radioproteção e substituto do chefe de divisão, o que 
o limita na disponibilidade de horários.

Apesar da experiência dos servidores historicamente envolvidos com o descomissionamento, não se sabe a motivação de ambos para tratar deste tema.


%\textbf{Última revisão:} \today

\chapter{Implementaçao de cálculos usando GPU no Serpent2}

\textbf{Motivação:} $\star\star\star\star\star$\\

\textbf{Tipo:} Projeto\\

\textbf{Descrição:} O objetivo deste projeto é o de fazer com que a atual 
implementação do Serpent2 (versão 2.1.30) seja capaz de fazer uso da capacidade de 
processamento de placas gráficas (GPU) para auxiliar nos cálculos de histórias de 
nêutrons.

\textbf{Sub-atividades:} (Em avaliação) \\

\begin{itemize}
	\item[1] Compreensão da forma como são feitos os cálculos de nêutrons na atual 
	versão do Serpent, primeiramente sequencialmente, depois em paralelo;
	\item[2] Descrição do(s) algoritmo(s) e gerenciamento de memória;
	\item[3] Implementação de protótipo utilizando OpenCL.
\end{itemize}

\textbf{Observações:}

Este software será uma modificação em software já existente cuja licença permite 
modificações para uso pessoal. Caso cheguemos a uma implementação viável, deverão 
ser estudadas formas de cessão do software para os autores originais.\\

\textbf{\textit{Outcomes}}

\begin{itemize}
	\item[1] Uma biblioteca específica registrável como produto tecnológico no 
INPI (sob condições já que é baseada num software restrito);
	\item[2] Publicação \textit{Annals of Nuclear Engineering} com exemplos de 
	utilização e comparação com versão sem GPU;
	\item[3] O conhecimento no uso de GPUs em computação científica é um diferencial, o que poderá trazer colaborações externas internacionais no âmbito de um posdoc;
	\item[4] O \textit{know-how} pode ser aplicável a outros projetos \ref{pyrana, pytera} do próprio LTHN que podem tirar proveito do novo cluster.
\end{itemize}

\textbf{Dificuldades/Restrições}

Pode ser impossível alcançar uma implementação em GPU mais eficiente do que 
a atual versão do Serpent2. Não está clara a forma de colaboração/cessão do software caso seja implementada uma versão mais rápida.\\

\chapter{Prospecção de POSDOC 2021}
\label{posdoc}

\textbf{Motivação:} $\star\star\star\star\star$\\

\textbf{Tipo:} Projeto\\

\textbf{Descrição:} Para propor, nos termos dos eventuais editais de pós-doutorado no exterior CAPES/CNPq, um projeto factível, executável e competitivo, é necessário 
desde já conectar as necessidades do projetos CDTN/CNEN com meus interesses técnicos juntamente com uma proposta representativa para a instituição anfitriã 
no exterior.

Tenho hoje dois caminhos a pensar para o posdoc:

\section{Proposição TRIGA}
\label{TRIGA}
% Trabalho em reatores do tipo TRIGA com vistas ao eventual descomissionamento do nosso reator TRIGA IPR-R1 juntamente com atividades de desenvolvimento computacional. Esta talvez seja uma proposta de difícil planejamento no tocante ao casamento das atividades. Uma atividade de descomissionamento pode ter pouco a se relacionar com uma atividade de desenvolvimento.
 
A ideia inicial de envolver descomissionamento foi abortada. Segue abaixo a lista inicial prospectiva de possíveis centros com reatores TRIGA:
 \begin{enumerate}
 	\item Universidade de Pavia, IT. Tem TRIGA, seu pouco das atividades deles.
 	\item Kansas State Univesity: Possuem um TRIGA e o prof. James publicou sobre acoplamento na mesma época que eu. É uma opção promissora para unir TRIGA/desenvolvimento. Entretanto, não é um deparamento de Eng. Nuclear importante nos EUA.
 	\item Slovenia: os contatos existem com o Daniel. TRIGA e muito trabalho em Eng. Nuclear. Além do mais, o CDTN tem um convênio assinado com eles, o que pode 
 	ser um facilitador ao submeter o projeto ao órgão de fomento.
 \end{enumerate}

\subsection{Kansas State University - KSU}

Acompanhando as atividades de pesquisa do professor Jeremy Roberts, foi publicado 
(aparentemente em 2018, via Research Gate) um trabalho sobre processos de predição de composição de combustíveis de reatores nucleares (\textit{Comparison Between Gaussian Processes and DMD Surrogates for Isotopic Composition Predicition}\cite{Roberts}).

O trabalho citado tem como objetivo estimar as atuais composições isotópicas 
do reator KSU TRIGA Mark II. Essa metodologia visa reduzir o tempo gasto em 
simulações e também visa atacar o problema de que eles não possuem o histórico 
detalhado de queima dos combustíveis avaliados. \textbf{Essa não é a mesma 
situação do CDTN}, cabe enfatizar. O combustível TRIGA é discretizado em 
três zonas axiais.

Apesar das diferenças entre KSU e CDTN, este trabalho pode servir de base 
para um projeto de pos-doc que seja também de interesse da KSU.

Atividades?
\begin{itemize}
\item Avaliar o momento de contactar o prof. Roberts;
\item Acrescentar neste documento os trabalhos similares aos meus.
\end{itemize}



\section{Proposição Computação científica}

Nesta proposta, o objetivo é desenvolver alguma \textit{feature} necessária a um 
dos softwares utilizados por nós ou voltar a atacar o problema acoplado, que 
continua de interesse da comunidade.

As dificuldades, aqui, estão numa justificativa robusta de modo a convencer o 
órgão de fomento que vale a pena apoiar tal proposta que não proponha experimentos 
no exterior.

Possíveis centros para Computação Científica:
\begin{enumerate}
	\item VTT, Finlândia. São os desenvolvedores do código de Monte Carlo, que utilizamos no LTHN e temos o código-fonte. Conheço pessoalmente parte da equipe e, 
	com uma proposta relativamente simples de desenvolvimento, seria fácil ser aceito por eles. O TRIGA da VTT foi descomissionado em 2016.
	\item NCSU, North Carolina State University, EUA. A Maria Avramova é manda-chuva aqui e o André a conhece pessoalmente. Me passou o e-mail. Andou trabalhando com Monte Carlo e se supõe que não seja difícil conseguir a assinatura dela pra autorizar minha ida pra lá acoplar códigos. É necessário iniciar os contatos. É uma universidade respeitada na área nuclear.
	\item Wisc, Madison, EUA. O departamento é relativamente conhecido e não conheço ninguém lá. Entretanto, é um dos pioneiros no desenvolvimento de software na área nuclear de forma colaborativa. Isso permite iniciar uma colaboração independente e depois criar conexões com os pesquisadores. Eles desenvolvem um simulador Monte Carlo e querem adaptá-lo ao Serpente. Isso só já seria um pós-doc. Olhar uma eventual implementação de Delta-Tracking no código deles. Referências: \url{http://cnerg.github.io/projects/main.html} e \url{http://svalinn.github.io/DAGMC/}. Eles publicam vagas sempre relacionadas a desenvolvimento. O contato, via site, é Paul Wilson (paul.wilson@wisc.edu), Professor, e líder do CNERG.
	
\end{enumerate}

\subsection{CNERG - WISC}
	Do ponto de vista de grupo de alto-nível, este é sem dúvida um dos principais. Realmente o trabalho é devotado a desenvolvimento. Acho que vale a tentativa de contato (a partir de julho de 2019).
	
	Dado o razoável número de projetos de software, gerenciados ou com participação do CNERG \url{http://cnerg.github.io/projects/software.html}, é fundamental avaliar quais os possíveis pontos de colaboração antes de estabelecer contato. Eventualmente, vale avaliar minha participação na contribuição antes de fazer contatos. Os seguintes softwares estão na página \textbf{svallin} \url{https://github.com/svalinn} e despertam interesse inicial:
	
\begin{enumerate}
	\item DAGMC: Direct Accelerated Geometry Monte Carlo Toolkit;
	\item MCNP2CAD: \textit{A program to extract the geomtery from MCNP input files and write it out using any ITAPS iGeom backend};
	\item DAGMC-viz: Um visualizador para o DAGMC.
\end{enumerate}

O DAMC parece promissor e vai ao encontro da minha proposta do "visualizador" Monte Carlo para MCNP e Serpent. Seria interessante tentar juntar as coisas no desenvolvimento, ao menos referenciar como as coisas são feitas no DAGMC.

\textbf{Observações:}
\\


\textbf{\textit{Outcomes}}

Realização de 1 ano de pós-doutorado no exterior. Eventual (e esperada) publicação 
de trabalhos científicos. Reforço ou início de cooperação internacional.

\textbf{Dificuldades/Restrições}

Definir um projeto consistente e que desperte o interesse da CAPES/CNPq ao mesmo 
tempo em que se justifique a estadia no exterior.




\end{document}