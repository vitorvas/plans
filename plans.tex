%Tudo que começa com '%' é comentário e é ignorado pelo compilador

%Gerando arquivo em latex:
%latex arquivo.tex (em dvi)
%pdflatex arquivo (em pdf)
%dvipdfm arquivo
%s2pdf arquivo

% Alguns modos de usar o latex:
% Windows – Miktex com Led
% Linux – texlive com kile

\documentclass[12pt]{report} %aqui fala o tipo de documento e o tamanho da fonte. Opções: tamanho do texto (10pt, 12pt, 14pt), formato do papel (a4paper, a5paper, b5paper, letterpaper, legalpaper, executivepaper), o número de colunas (onecolumn, twocolumn), entre outras opções.
%Por exemplo, [12pt,a4,twocolumn].
%classe: article, report, letter, book ou slides. Instalar abnt para quem está pensando no tf
\usepackage[brazilian]{babel} %hifenização em português do brasil
\usepackage[T1]{fontenc} % caracteres com acentos são considerados um bloco só
\usepackage[utf8]{inputenc} % Corrigie os acentos em Português
\usepackage{ae} %arruma a fonte quando usa o pacote fontenc
%\usepackage[pdftex]{graphicx}%Para inserir figuras
\usepackage{vhistory}

\begin{document}
\title{Projetos e atividades}
\author{Vitor Vasconcelos Araújo Silva\\ vitors@cdtn.br}
\date{\today}

\maketitle %cria o título

%\def \negritovi {\textbf} %Criando comandos

\tableofcontents %índice
\pagebreak % Quebra de página
%\listoffigures %indice de figuras
%\listoftables %indice de tabelas
\pagebreak % Quebra de página

\begin{abstract}
Este documento apresenta os projetos e atividades planejados e/ou execução 
dentro da minha área de atuação no SETRE/CDTN. Para cada um deles são 
brevemente apresentados objetivo, motivação, aplicações, possíveis \textit{outcomes} e nível de dificuldade de execução. O objetivo da 
condensação das atividades e projetos em um documento é o de simplificar 
a avaliação temporal da evolução dos planos e organizar o tempo investido 
em cada um deles.
\end{abstract}

\begin{versionhistory}
	\vhEntry{0.0}{21/11/2018}{Vitor}{Documento criado}
%	\vhEntry{1.1}{23.01.04}{DP|JPW}{correction}
%    \vhEntry{1.2}{03.02.04}{DP|JPW}{revised after review}
\end{versionhistory}

\chapter{Visualizador de geometrias Serpent}

\textbf{Motivação:} $\star\star\star\star\star$\\

\textbf{Tipo:} Projeto\\

\textbf{Descrição:} O objetivo deste projeto é de construir um software capaz 
de ler um arquivo de entrada do Serpent e gerar um modelo visual 3D da geomtria 
descrita. O Serpent utiliza primariamente métodos de geometria sólida construtiva 
(CSG) separada em universos para definir regiões para cálculo neutrônico. A 
contribuição esperada é permitir que os usuários do Serpent possam visualizar 
sua entrada em busca de partes irregulares ao, mesmo, visualizar partes internas 
ou elementos de interesse.

Não estão previstas inicialmente modificações na visualização como cortes em 
planos ou filtrons afins. A ideia é, uma vez gerados os elementos descritos no 
input, além de visualizá-los no monitor, gravar os elementos em um arquivo 
do formato VTK. Este arquivo, por sua vez, pode ser nativamente aberto pelo 
Paraview, que oferece diversos tipos de formas de manipulação dos dados para 
visualização.

Numa segunda etapa é interessante extender as classes relativas à leitura dos elementos Serpent para o formato MCNP, aumentando a aplicabilidade do software 
já que a base de usuários do MCNP é consideravelmente maior.\\

\textbf{Sub-atividades:} (Em avaliação) \\

%\begin{itemize}
%	\item[1] 
%	\item[2]
%	\item[3]
%\end{itemize}

\textbf{Observações:}

Este software será construído utilizando a biblioteca \textit{VTK}. Esta biblioteca 
é originalmente escrita em C e C++, mas possui \textit{bindings} para Python. Ainda não está decidida qual linguagem será utilizada na sua construção. Protótipos 
iniciais estão sendo feitos em Python.\\

\textbf{\textit{Outcomes}}

\begin{itemize}
	\item[1] Um software completo e registrável como produto tecnológico no 
INPI;
	\item[2] Publicação \textit{Journal of Open Source Software};
	\item[3] Publicação \textit{Annals of Nuclear Engineering} com exemplos de 
	utilização.
\end{itemize}

\textbf{Dificuldades/Restrições}

Para a publicação no JOSS, não basta o software com código aberto, mas é necessária 
documentação formal e em formato aberto, bem como testes de software também formais.\\

 %Cria uma seção
\chapter{Novo cluster: instalação e gerência}

\textbf{Motivação:} $\star\star\star\star\star$\\

\textbf{Tipo:} Projeto que se tornará atividade.\\

\textbf{Descrição:} Este projeto trata da instalação do novo cluster do Laboratório 
de Termohidráulica e Neutrônica do CDTN (LTHN). Este sistema compreende uma máquina mestra e 8 máquinas escravas, todas equipadas com placas gráficas Nvidia que podem ser utilizadas tanto para visualização quanto para cálculos em paralelo.

O sistema está parcialmente instalado com Linux CentOS 7.5 com um serviço de 
sistema de arquivos distribuído Gluster ativo. 

O objetivo final deste projeto é ter o cluster funcionando com usuários com cotas 
de disco e com os seguintes sistemas instalados:

\begin{itemize}
	\item \textbf{Serpent2 Monte Carlo [INSTALADO]};
	\item \textbf{MCNP 6 [INSTALADO]};
	\item \textbf{OpenFOAM versão 6 [INSTALADO]};
	\item \textbf{Pacote ANSYS [INSTALADO]};
	\item SCALE;
	\item \textbf{Matlab [INSTALADO (aguardando testes do Prana)]}.
\end{itemize}

No presente momento, o sistema NAS está instalado em modo padrão, com duas 
interfaces de rede 10Gbits conectadas as switch do cluster (endereços IP:\texttt{nas1: 13.13.13.51} e \texttt{nas2: 13.13.13.52}) e uma interface 
1Gb conectada à rede do CDTN. A arquitetura a ser usada no NAS ainda está sendo 
estudada, mas já foi configurado um volume em RAID 10 para o backup das máquinas 
do LTHN.

A autenticação é feita tanto com chaves públicas e privadas com SSH mas também 
está instalado o serviço RSH para autenticação sem criptografia. Isso funciona 
apenas internamente, entre máquinas do cluster e o master. O método 
\textit{hostbased authentication} falhou nas tentativas de configuração e, portanto, foi abandonado até um segundo momento, sendo utilizadas as duas 
formas acima de autenticação.\\

\textbf{Sub-atividades}

A partir da situação atual do cluster, as sequintes atividades estão previstas:
\begin{itemize}
%	\item[1] \textbf{Modificação da forma de autenticação dos hosts de chaves 
%	para autenticação no host}. Isso é fundamental para simplificar a criação 
%	de contas de usuários e diminuir o volume de dados de chaves armazenadas em 
%	um diretório \texttt{/home} compartilhado\footnote{"As configurações do \textit{host based authentication} falharam. Mais esforços serão empregados até um limite de tempo razoável. No entanto, caso o problema persista, será utilizada a solução atual com chaves públicas e primárias por usuário. Esta solução torna mais custosa a criação de contas de usuários, mas está funcional.};
	\item[1] Instalação e testes dos softwares descritos (falta apenas o SCALE);
	\item[2] Instalação do TORQUE ou Slurm\footnote{O Torque é o sistema padrão so CentOS. Entretanto, na literatura o Slurm é mais utilizado e será a provável escolha para o nosso cluster.} para lançamento de aplicações e controle de 
	carga de trabalho do sistema;
	\item[4] Implementação de capacidade de visualização remota pelas placas gráficas a partir de outras máquinas Linux do LTHN.
	\item[5] Avaliação da viabilidade de um hub para exploração dos sistemas de segurança do cluster (power supplies e no-break).
	\item[6] Avaliação de implantação de desligamento automático em casa de falta de energia atráves de conexão lógica com o sistema de controle do no-break.
\end{itemize}

As sub-atividades listadas são atualizadas a cada revisão no documento, sendo que 
as sub-atividades finalizadas são removidas da lista e, de acordo com a demanda, 
são acrescidas novas tarefas. Por exemplo, a instalação do NAS ou a instalação do 
matlab, que não estavam no projeto inicial do cluster.\\

\textbf{Observações:}

O processo de instalação pode levar a demandas não previstas, de acordo com as avalições feitas para implantação de determinada \textit{feature} no sistema.

Uma vez instalado e testado, o projeto passa a ser uma atividade. Esta atividade 
consiste na manutenção e verificação do funcionamento do sistema.\\

\textbf{\textit{Outcomes}}

\begin{itemize}
	\item[1] Sistema de cálculos para as atividades do LTHN e, eventualmente, para usuários externos.
	\item[2] Paper INAC 2019 com descrição do processo final de instalação e exemplos de aplicações com tempo de execução.

\end{itemize}


\textbf{Dificuldades/Restrições}

O sistema está hoje fora da rede elétrica do gerador.\\

Em 13/12/2018 um dos equipamentos de ar-condionado da sala do cluster falhou. A equipe de manutenção do CDTN foi acionado no mesmo dia.\\

É necessário solicitar a instalação de uma tomada 110V para a ligação do hub 
(disponibilizado pelo Élcio) e dos equipamentos de rede que vão necessitar 
dessas conexãos.\\

Última modificação: \date{\today}

%\date{\today}

\chapter{OpenFOAM solver \texttt{chtMultiRegionFoam}}

\textbf{Motivação:} $\star$\\

\textbf{Tipo:} Projeto\\

\textbf{Descrição:}

Este projeto visa modificar o solver \texttt{chtMultiRegionFoam} do 
OpenFOAM versão 6 para utilizar internamente um termo-fonte nas equações 
de energia.

Apesar de ser possível (e já termos feito) uso de termo-fonte através do 
metódo \textit{fvOptions} do OpenFOAM, acreditamos ser mais interessante 
modificar o solver. São modificações que já foram feitas anteriormente 
para a versão 2.2. Além disso, com o termo-fonte interno é mais facilmente 
visualizável o resultado pelo paraview.

O objetivo desse projeto é ter um solver atualizado para ser usado no acoplamento 
OpenFOAM+Serpent a ser (provavelmente) feito como tema de dissertação de mestrado 
do Tiago.\\

\textbf{Sub-atividades:}

\begin{itemize}
	\item[1] Modificar o arquivo \texttt{createSolidFields.H} para criar o volScalar para armazenar o termo-fonte; 
	\item[2] Modificar o arquivo \texttt{solveSolid.H} para incluir o termo-fonte na equação de energia;
	\item[3] Verificar se são necessárias modificações no arquivo \texttt{createSolidMeshes.H} para leitura do campo scalar para o termo-fonte.
\end{itemize}

\textbf{Observações:}

A modificação deve ser feita a partir da criação em separado de um solver 
(\texttt{chtMultiRegionSourceFoam}) pela cópia dos arquivos em um diretório qualquer.

Os arquivos de compilação utilizados pelo wmake devem refletir essas modificações.

É importante, para além da modificação do solver, ser capaz de fazer convergir os 
casos de testes originais do OpenFOAM e o caso adaptado da minha tese (disponível 
no github em: https://github.com/vitorvas/teste5)

Na máquina caprara-lx: \$FOAM\_APP/solvers/heatTransfer/chtMultiRegionFoam\\

\textbf{\textit{Outcomes}}

Esta modificação é muito simples para gerar um programa registrável. Entretanto, 
num cenário com scripts para a execução acoplada da dissertação do Tiago, isso 
pode ser viável.\\

\textbf{Dificuldades/Restrições}

Não se espera nada além das mesmas dificuldades da criação do solver para minha 
tese. \\

%\textbf{Última revisão:} \today

\chapter{\textcolor{red}{FINALIZADO}: Paralelização do \textit{milonga}}

\textbf{Motivação:} $\star\star$\\

\textbf{Tipo:} Projeto\\

\textbf{Descrição:}

\textbf{Aparentemente o milonga será incorporado a um novo código chamado Fino.
  Importante ficar atento ao trabalho do Germán neste projeto até 2023.}

Este projeto objetiva a paralelização do código milonga. O milonga é um solver 
de elementos finitos e volumes finitos para a solução da equação de difusão de 
neutrons. Funciona sobre o código wasora, que oferece funções de base sobre as 
quais o milonga executa.

A ligação intrínseca entre ambos os softwares, torna complexa a paralelização. 
Uma primeira metodologia de paralelização consiste em separação de domínios. 
Em outras palavras, a malha é dividida entre processadores e nas faces separadas 
são definidas condições de contorno especiais nas quais a comunicação entre 
processadores pode ser feita (utilizando, por exemplo, MPI).

Uma formulação mais simples para a paralelização consiste na solução das matrizes 
já construídas, seja por elementos finitos ou volumes finitos, em paralelo. Em 
teoria isso é facilmente atingido uma vez que o milonga faz uso da biblioteca 
PETSc para essa solução e esta biblioteca é construída utilizando-se do MPI. 
Apesar de mais simples se comparada à proposta anterior, uma vez que são feitas 
chamadas espalhadas entre o milonga e o wasora de funções da PETSc para a construção das matrizes, uma paralelização nesse nível é não-trivial.

Em resumo, quaisquer opções são bastante trabalhosas em termos de investimento 
de tempo em re-codificação.

Cabe ressaltar que já foram feitos alguns estudos com profiling do milonga para 
diferentes casos e algumas funções que percorrem a malha são os \textit{bottlenecks} conhecidos. Modificações na malha para tratar deste problema 
são também complexas.\\

\textbf{Sub-atividades:}

\begin{itemize}
	\item[1] Separação da construção das matrizes com execução do código comum 
	apenas no processo mestre.
\end{itemize}

\textbf{Observações:}

Problema muito interessante do ponto de vista didático, para a compreensão 
de FEM e FVM, neutrônica e como são feitas as soluções. Entretanto, bastante 
complexa qualquer modificação no código.

O milonga e o wasora são ambos construídos unicamente usando C e são opensource.\\

\textbf{\textit{Outcomes}}

Código paralelizado (não sei se cabe registro), eventual paper sobre a parelização 
executando casos complexos numa revista como Annals, ganho de know-how para eventual futuro pós-doc e reconhecimento de uma pequena comunidade utilizadora 
do milonga.\\

\textbf{Dificuldades/Restrições}

Tempo e esforço envolvidos com risco alto de não ser viável a paralelização.\\

\textbf{Resultados}

Vários testes com exemploes do \textit{milonga} para malhas com diferentes números de elementos mostraram, para o método de difusão por volumes finitos, que grande parte 
do tempo de computação (cerca de $44\%$) é gasto em funções de percurso da malha.

Apenas uma máximo de $10\%$ do tempo é gasto em operações de funções do módulo 
de difusão. 

A finalização ideal deste projeto seria com um relatório formal de \textit{benchmark} de três diferentes tipos de malhas (em número de elementos) 
para cada tipo de formulação:

\begin{enumerate}
\item Difusão volumes finitos;
\item Difusão elementos finitos;
\item Ordenadas discretas ($S_N$) volumes finitos;
\item Ordenadas discretas ($S_N$) elementos finitos;
\end{enumerate}

Devido à minha saída de licença, esse relatório não seja realizado já que não há 
potencial de publicação. Entretanto, permanece o interesse didático, em especial no entendimento das formulações e no uso das bibliotecas PETSc e SLEPc.


\end{document}