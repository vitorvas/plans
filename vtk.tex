\chapter{Visualizador de geometrias Serpent}

\textbf{Motivação:} $\star\star\star\star\star$\\

\textbf{Tipo:} Projeto\\

\textbf{Descrição:} O objetivo deste projeto é de construir um software capaz 
de ler um arquivo de entrada do Serpent e gerar um modelo visual 3D da geomtria 
descrita. O Serpent utiliza primariamente métodos de geometria sólida construtiva 
(CSG) separada em universos para definir regiões para cálculo neutrônico. A 
contribuição esperada é permitir que os usuários do Serpent possam visualizar 
sua entrada em busca de partes irregulares ao, mesmo, visualizar partes internas 
ou elementos de interesse.

Não estão previstas inicialmente modificações na visualização como cortes em 
planos ou filtrons afins. A ideia é, uma vez gerados os elementos descritos no 
input, além de visualizá-los no monitor, gravar os elementos em um arquivo 
do formato VTK. Este arquivo, por sua vez, pode ser nativamente aberto pelo 
Paraview, que oferece diversos tipos de formas de manipulação dos dados para 
visualização.

Numa segunda etapa é interessante extender as classes relativas à leitura dos elementos Serpent para o formato MCNP, aumentando a aplicabilidade do software 
já que a base de usuários do MCNP é consideravelmente maior.\\

\textbf{Sub-atividades:} (Em avaliação) \\

%\begin{itemize}
%	\item[1] 
%	\item[2]
%	\item[3]
%\end{itemize}

\textbf{Observações:}

Este software será construído utilizando a biblioteca \textit{VTK}. Esta biblioteca 
é originalmente escrita em C e C++.\\%, mas possui \textit{bindings} para Python. Ainda não está decidida qual linguagem será utilizada na sua construção. Protótipos 
%iniciais estão sendo feitos em Python.\\
Durante o ano de licença (2020) codifiquei alguns protótipos simples usando C++. Foram experimentadas as classes a serem usadas e uma forma inicial de geração
dos sólidos. No protótipo final, um arquivo vtk é escrito a partir de geometria simples pré-definida. Resultados encorajadores.\\

No mesmo ano de 2020 a equipe da \href{https://vtk.org/}{Kitware} atualizou toda a documentação do \textit{VTK}.

\textbf{\textit{Outcomes}}

\begin{itemize}
	\item[1] Um software completo e registrável como produto tecnológico no 
INPI;
	\item[2] Publicação \textit{Journal of Open Source Software};
	\item[3] Publicação \textit{Annals of Nuclear Engineering} com exemplos de 
	utilização.
\end{itemize}

\textbf{Dificuldades/Restrições}

Para a publicação no JOSS, não basta o software com código aberto, mas é necessária 
documentação formal e em formato aberto, bem como testes de software também formais.\\

